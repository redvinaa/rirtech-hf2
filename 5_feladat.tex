\section{Szögsebesség szabályozás állapotvisszacsatolással}

%{{{ Pólusok számítása
\subsection{Pólusok számítása}

Írjuk át a rendszerünket állapotteres alakra, valamint a $\fn{W}_\text{c}$ kontrollert
cseréljük le a $\mathrm{K}_\text{x}$ és $\mathrm{K}_\text{r}$ állapotvisszacsatoló és
alapjelkompenzáló mátrixokra.
A $\mathrm{D}$ kimeneti mátrix esetünkben zérus, és egyenlőre az
alapjelkompenzáló mátrix egységnyi.

\begin{figure}[H]
    \centering
    \incfig[.8\textwidth]{5a_allapotter_hatasvazlat}
    \caption{Állapottér modell hatásvázlata}
    \label{fig:5a_allapotter_hatasvazlat}
\end{figure}

A rendszer átviteli függvény formában van megadva:
\begin{equation}
	\fn{W}_\text{p} = \frac{\Psi}{(1+T_1s)(1+T_2s)} = \frac{\Psi}{T_1T_2s^2+(T_1+T_2)s+1},
\end{equation}
amelynek a pólusai $p_1=-\frac{1}{T_1} = -68,97$ és $p_2=-\frac{1}{p_2} = -7233,3$.

Írjuk át ezt állapottér modellre, irányíthatósági kanonikus alakra:
\begin{equation}
	\fn{Y} = \frac{\Psi}{T_1T_2s^2+(T_1+T_2)s+1}\fn{U}
\end{equation}
\begin{equation}
	\fn{X} := \frac{}{T_1T_2s^2+(T_1+T_2)s+1}\fn{U}\,\Rightarrow
\end{equation}
\begin{equation}
	u = T_1T_2\ddot{x} + (T_1+T_2)\dot{x} + x\,\Rightarrow
\end{equation}
\begin{equation}
	\underbrace{\mat{\dot{x}\\\ddot{x}}}_{\dot{\B{x}}}
	= \underbrace{\mat{0&1\\\frac{1}{T_1T_2}&\frac{T_1+T_2}{T_1T_2}}}_{\B{A}}
	\underbrace{\mat{x\\\dot{x}}}_{\B{x}}
	+ \underbrace{\mat{0\\1}}_{\B{B}}u
\end{equation}
\begin{equation}
	y = \underbrace{\mat{\Psi&0}}_{\B{C}}\mat{x\\\dot{x}} + \underbrace{\mat{0}}_{\B{D}}u
\end{equation}

A feladatkiírás alapján a domináns időállandó $\widetilde{T}_1=50~\text{ms}~=0,05$ s
kell hogy legyen. A másik pedig valami.%TODO

A visszacsatolás miatt
\begin{equation}
	u = \B{K}_\text{r}\B{r} - \B{K}_\text{x}\B{x}
\end{equation}
\begin{equation}
	\dot{\B{x}} = \B{A}\B{x} + \B{B}\B{K}_\text{r}\B{r} - \B{B}\B{K}_\text{x}\B{x}
\end{equation}
\begin{equation}
	\dot{\B{x}} = \brc{\B{A}-\B{B}\B{K}_\text{x}}\B{x} + \B{B}\B{K}_\text{r}\B{r}
\end{equation}
Ebben a feladatrészben $\B{K}_\text{r} \equiv \B{I}$, tehát a referencia bemenetre nézve
a $\B{B}$ bemeneti mátrix nem változik, az új rendszermátrix pedig $\widetilde{\B{A}} = \B{A}-\B{B}\B{K}_\text{x}$.
A $\B{K}_\text{x}$ mátrixot úgy kell megválasztani, hogy az új rendszermátrix sajátértékei
a kívánt pólusok legyenek.

\begin{equation}
	\widetilde{\B{A}} = \mat{0&1\\\frac{1}{T_1T_2}&\frac{T_1+T_2}{T_1T_2}} - 
	\mat{0&0\\k_1&k_2} = \mat{0&1\\\frac{1}{T_1T_2}-k_1&\frac{T_1+T_2}{T_1T_2}-k_2}
\end{equation}
\begin{equation}
	\det\widetilde{\B{A}} = k_1 - \frac{1}{T_1T_2} = 0\,\Rightarrow k_1=\frac{1}{T_1T_2}
\end{equation}

%}}}
