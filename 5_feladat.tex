\section{Szögsebesség szabályozás állapotvisszacsatolással}

%{{{ Pólusok számítása
\subsection{Pólusok számítása}

Írjuk át a rendszerünket állapotteres alakra, valamint a $\fn{W}_\text{c}$ kontrollert
cseréljük le a $\mathrm{K}_\text{x}$ és $\mathrm{K}_\text{r}$ állapotvisszacsatoló és
alapjelkompenzáló mátrixokra.
A $\mathrm{D}$ kimeneti mátrix esetünkben zérus, és egyenlőre az
alapjelkompenzáló mátrix egységnyi.

\begin{figure}[H]
    \centering
    \incfig[.8\textwidth]{5a_allapotter_hatasvazlat}
    \caption{Állapottér modell hatásvázlata}
    \label{fig:5a_allapotter_hatasvazlat}
\end{figure}

A rendszer átviteli függvény formában van megadva:
\begin{equation}
	\fn{W}_\text{p} = \frac{\Psi}{(1+T_1s)(1+T_2s)} = \frac{\Psi}{T_1T_2s^2+(T_1+T_2)s+1},
\end{equation}
amelynek a pólusai $p_1=-\frac{1}{T_1} = -68,97$ és $p_2=-\frac{1}{p_2} = -7233,3$.

Írjuk át ezt állapottér modellre, irányíthatósági kanonikus alakra:
\begin{equation}
	\fn{Y} = \frac{\Psi}{T_1T_2s^2+(T_1+T_2)s+1}\fn{U}
\end{equation}
\begin{equation}
	\fn{X} := \frac{}{T_1T_2s^2+(T_1+T_2)s+1}\fn{U}\,\Rightarrow
\end{equation}
\begin{equation}
	u = T_1T_2\ddot{x} + (T_1+T_2)\dot{x} + x\,\Rightarrow
\end{equation}
\begin{equation}
	\mat{\dot{x}\\\ddot{x}} = \mat{0&1\\\frac{1}{T_1T_2}&\frac{T_1+T_2}{T_1T_2}}\mat{x\\\dot{x}}
	+ \mat{0\\1}u
\end{equation}
\begin{equation}
	y = \mat{\Psi&0}\mat{x\\\dot{x}} + \mat{0}u
\end{equation}

A feladatkiírás alapján a domináns időállandó $\widetilde{T}_1=50~\text{ms}~=0,05$ s
kell hogy legyen. A másik pedig valami.%TODO


%}}}
